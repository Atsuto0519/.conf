% Intended LaTeX compiler: pdflatex
\documentclass[10.5pt, aJ4]{jarticle}
               
\usepackage{$HOME/.emacs.d/latex/fancyhdr}
\usepackage{$HOME/.emacs.d/latex/proceeding}
\makeatletter
\long\def\@makecaption#1#2{%
  \vskip\abovecaptionskip  \iftdir\sbox\@tempboxa{#1\hskip1zw#2}%
    \else\sbox\@tempboxa{#1~ #2}%
  \fi
  \ifdim \wd\@tempboxa >\hsize
    \iftdir #1\hskip1zw#2\relax\par
    \else #1~ #2\relax\par\fi
  \else
  \global \@minipagefalse
  \hbox to\hsize{\hfil\box\@tempboxa\hfil}%
\fi
\vskip\belowcaptionskip}
\def\WordCount#1{%
  \@tempcnta\z@
  \@tfor \@tempa:=#1\do{\ignorespaces \advance\@tempcnta\@ne}%
  #1
  \\ \hrulefill \\
  \vspace{-8mm}
  \begin{flushright}
    {\bfseries 文字数: \the\@tempcnta 文字}
  \end{flushright}
}
\makeatother
\author{あざ◎}
\date{\today}
\title{現代文明レポート}
\begin{document}

\maketitle
\tableofcontents
\vspace{10mm}


\section{古代ローマ世界(第10回)}
\label{sec:org199beb3}

古代ローマといえば,「すべての道はローマに通ず」と例えられるほど強大な国家として君臨していた.
紀元前247年,現在のチュニジアの首都であるカルタゴと呼ばれる国家にハンニバル・バルカは生まれる.
ハンニバルが生まれる数年前に第一次ポエニ戦争が行われており,この戦争ではシチリアの派遣を巡ってカルタゴがシチリアに攻め込んだが
ローマとシチリアは同盟を結んでいたため,カルタゴはローマによる征服の危機を迎えていた \cite{Hannibal+NationalGeographic} .
ローマは地中海の貿易で栄えた海洋国家という側面を持ち,地中海を挟んだ両国の戦いはローマが有利だった.
そのためカルタゴは地中海沿岸戦に手を焼き,シチリア,コルシカ,サルディニアの島々を明け渡した.

時代は進み22歳になったハンニバルは,元々将軍だった父ハミルカルの死後からしばらくの後,
スペインにおけるカルタゴ軍の司令官に選出された.
司令官になったハンニバルは,これまでと打って変わり陸路を利用してイタリア半島へ,非常に大胆な奇襲を計画した.
まずハンニバル率いるカルタゴ軍は地中海の西へ向かい行進を開始した.
ハンニバルらは地中海をまたげばすぐにイタリア半島へ乗り込むことができる地形に対して,
地中海沿岸を時計回りに沿って移動する陸路を選んだ.
当時は当然徒歩しか移動手段はなく,陸路で移動するには非常に困難であるため,
ローマはカルタゴがこのような奇襲を計画するとは夢にも思っていなかった.
結果として,カルタゴ軍はアルプス山脈越えを決行したために多くの犠牲を出したが,
イタリア半島新入は成功し優勢と思われていたローマを恐怖のどん底へ陥れた.
この奇襲が第二次ポエニ戦争の引き金となるが,その後のカルタゴ軍はローマに敗戦を期する結果となった.

ハンニバルが攻め込んだ時代から更に進み,ローマは紀元前100年ほどになると内乱の時代を迎えることになる.
ローマの内乱が勃発した原因は,現在のトルコの北部に位置するポントスの王ミトリダテスがローマの属国を侵略したため,
オリエント遠征の指揮官をマリウスとスッラのどちらにするかとの抗争が発端になった.
最終的にはスッラが独裁官となり覇権を握ることになり,
マリウスの甥に当たるガイウス・ユリウス・カエサルはスッラに目をつけられローマにいられなくなった.
カエサルはローマから離れ留学を決意するが,逃亡中にスッラの部下に捉えられてしまうがのちに亡命も兼ねて海外留学した \cite{Caesar+Y-History} .
スッラが亡くなりしばらくするととカエサルはローマへ帰還し,弁舌と財力で財務官,按察官.大神官に任命された.



\section{古代ギリシア文明とアレクサンドロス大王(第11回)}
\label{sec:org049c5c7}

マケドニア王国はギリシア北方の大国で,紀元前359年に彼の有名なアレクサンドロス大王の父である
フィリッポス2世が22歳で王位につく \cite{Alexandros+Y-History,Alexandros+Uraken} .
このときまでマケドニア王国はギリシアの僻地に属し文化などで馬鹿にされていたが,
フィリッポス2世は国政と軍政の改革を進め国力をつけ侵略を開始する.
紀元前338年のカイロネイアの戦いで強国アテネと一時覇権を握ったテーバイの連合軍を破り勝利した後に,
ギリシアの各ポリスとコリントス同盟を結びマケドニアの支配下に置く.
このときスパルタは結ばず,テーバイは徹底的に破壊された.

イスカンダルとは,古代マケドニア王国のアレクサンドロス大王を指すアラビア語・ペルシア語の人名である.
紀元前336年にフィリッポス2世は暗殺されてしまい,当時20歳であったアレクサンドロス3世,
後のアレクサンドロス大王が王位を継承する.
アレクサンドロスとは英語読みであり,イスカンダルと同じ意味の単語である.
フィリッポス2世はアレクサンドロスに対して徹底的な教育を施し,
アテネから招かれた哲学者アリストテレスを家庭教師につける.
アレクサンドロスは大王や軍人の功績が語られることが多いが,
アリストテレスの熱心な生徒としてギリシア文化を吸収するなど没するまで学問を好んだという.
彼はコリントス同盟の盟主としてのマケドニア王位につき,同盟国に対して統制する立場でもあった.
まず彼は北方のドナウ川方面を平定し,また離反しようとするテーバイを討ち,
各同盟国を統率して紀元前334年にペルシア帝国を始めとする東方遠征を開始した.
同年,アレクサンドロスはアケメネス朝ペルシアにマケドニア兵,同盟国からなる重騎兵,軽騎兵,長槍兵の軍隊で行軍する.
快進撃を続け,翌年の紀元前333年にイッソスの戦いでダレイオス3世率いるペルシア軍を打ち破る.
更に紀元前332年,エジプト遠征も成功させナイル川河口にアレクサンドリアという大都市を建設し,
紀元前330年にダレイオス3世は部下によって殺されペルシア帝国を滅ぼした.
アレクサンドロスはギリシアから古代オリエント世界を含む帝国を出現させ,
エジプトではファライトして,ペルセポリスではアケメネス朝後継者として自身を神格化した.
しかしこのふるまいがマケドニア人やギリシア人の反感を買い,
進軍を続け中央アジアからインドへ向けてインダス川上流を越えたが,
アレクサンドロスの侵略計画は部下の反対によって実現が叶わなかった.


\bibliographystyle{junsrt}
\bibliography{reference}
\end{document}